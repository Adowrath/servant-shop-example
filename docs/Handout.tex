\documentclass[11pt,utf8,english,a4paper]{article}

\usepackage[english]{babel}
\usepackage[utf8]{inputenc}
\usepackage{csquotes}
\usepackage[left=2.5cm,right=2.5cm]{geometry}   % kleinere Ränder
\usepackage{microtype}
\usepackage[table]{xcolor}
\usepackage{pgfplots}
\pgfplotsset{compat=1.18}
\usepackage{graphicx}
\usepackage{multicol}
\usepackage{array}

\usepackage{parskip}
\usepackage{comment}
\usepackage[orig,english]{isodate}
\usepackage[bottom]{footmisc}
\usepackage{fancyhdr}
\usepackage[T1]{fontenc}
\usepackage{lmodern}

\usepackage{float}
\floatstyle{plain}
\restylefloat{figure}
\usepackage{minted}
\setminted[]{frame=leftline, breaklines, breakbytoken, encoding=utf8, autogobble=true, fontsize=\footnotesize}
\usemintedstyle{xcode}
\makeatletter
\renewcommand{\fps@listing}{htp}
\makeatother

\usepackage{tikz}
\usetikzlibrary{matrix, positioning}

\usepackage[title]{appendix}

\usepackage[hidelinks]{hyperref}
\urlstyle{same} % gleiche Schriftart für Hyperlinks.
\hypersetup{
  colorlinks,
  linkcolor={red!50!black},
  citecolor={blue!50!black},
  urlcolor={blue!80!black}
}

% keine Widows und Orphans
\widowpenalty10000
\clubpenalty10000

\pagestyle{fancy}
\fancyhf{}
\setlength{\headheight}{25pt}
\renewcommand{\footrulewidth}{0.4pt}
\chead{Servant - Handout}
\rhead{}
\lfoot{Cyrill Brunner}
\cfoot{}
\rfoot{\thepage}


\begin{document}

\section*{Servant - Typesafe DSL for describing \& deriving Servers and Clients}

Servant is a type-level DSL written in the functional programming language Haskell.
It has a focus on encoding as much as possible about each endpoint of a server
on the type-level, thus enabling automatic derivation of server-side handler type signatures
and -- more impressively -- fully functional client-side accessors.

Because your API is statically known, it is also incredibly simple to generate documentation
for it -- be it a custom layout, or via a defined standard like Swagger/OpenAPI.

At the same time, Servant is designed to be incredibly extensible and pluggable.
Almost all aspects of its system can be extended ad-hoc -- a new return type? Define (or
more realistically, derive a new typeclass instance) so it can convert your value
and send it in a Response.

Want to require a whole section of your server require authentication? No need to declare
that separately from the API definition itself -- just include the combinator in the API.
Not content with the existing Auth combinator not being able to verify login tokens via
your database? Just define a new combinator that sits on top and that queries your database,
deciding to terminate the routing process before it even reaches your handler.

A good start to Servant is given through its \enquote{Read the Docs} page at \url{https://docs.servant.dev/en/stable/tutorial/index.html}.
The Tutorial gives a nice overview over the core batteries-included functionalities in
Servant, while the Cookbook takes a more broad look at the ecosystem of available pluggable
libraries.

As mentioned, Servant is written in Haskell, a pure, lazy, functional programming language.
If one is mainly acquainted with languages like C++, Java, C\#, Python, JavaScript and the like,
Haskell might provide an interesting challenge to tackle and to remold how one thinks about
programming.  As a pure language, Haskell does not directly offer any mutation -- all is handled
via constructing updated copies of basic datastructures.
To enable efficiency while also encapsulating the side-effects of mutability, or interaction
with outside concepts like the filesystem, such operations are constrained into a specific
context called \texttt{IO}.
But it is possible to define more specific \enquote{effects} as well -- while \texttt{IO}
is a blunt hammer that just screams \enquote{I'm Impure!}, \texttt{State} can be used to
pass along a mutable state in your application, \texttt{Except} can model throwing and catching
exceptions, while \texttt{STM} -- short for \enquote{Software Transactional Memory} allows for
deadlock- and race condition-free concurrent programming.

Servant itself only concerns the design and implementation of your servers API -- what URLs with
which request bodies, headers, authentications etc. can the server handle?
If you need to use a database -- and most realistically, you do -- you simply choose a library for
it, construct the connection at server startup and then make it available by tweaking the
type of your handlers once -- making the connection available as a \texttt{Reader(T)} context.
From then on, you can simply use the connection with the libraries functions as if they
were a built-in concept.
One popular such library is \texttt{persistent}, which also functions as an ORM both to SQL and NoSQL-databases.
And just like Servant, it is completely typesafe.
For more advanced queries, the SQL-specific library \texttt{esqueleto} offers a DSL for writing queries.

And last but not least, the obligatory small example: A small counter server keeping track of the last
modification.

\begin{minted}{haskell}
  {-# LANGUAGE DataKinds, DeriveGeneric, TypeOperators, BlockArguments #-}
  {-# LANGUAGE DeriveAnyClass, RecordWildCards, DerivingStrategies #-}
  module Main where
  
  import Control.Monad.IO.Class
  import Data.Aeson
  import Data.IORef
  import Data.Proxy
  import GHC.Generics
  import Network.Wai.Handler.Warp
  import Servant.API
  import Servant.Server
  
  data Counter = Counter { count :: Int, lastModification :: Int }
    deriving stock (Eq, Show, Generic)
    deriving anyclass (FromJSON, ToJSON)
  
  type API =
    -- GET /count, returns { "count": 20, "lastModification": 0 }
    "count" :> Get '[JSON] Counter
    -- POST /add with Int as JSON, returns new Counter
    :<|> "add" :> ReqBody '[JSON] Int :> Post '[JSON] Counter
    -- POST /subtract with Int as JSON, returns nothing
    :<|> "subtract" :> ReqBody '[JSON] Int :> NoContentVerb 'POST
  
  getCount :: IORef Counter -> Handler Counter
  getCount ref = liftIO $ readIORef ref
  
  addCount :: IORef Counter -> Int -> Handler Counter
  addCount ref i = do
    Counter {..} <- liftIO $ readIORef ref
    let counter = Counter { count = count + i, lastModification = i }
    liftIO $ writeIORef ref counter
    return counter
  
  subCount :: IORef Counter -> Int -> Handler NoContent
  subCount ref i = do
    liftIO $ modifyIORef ref \Counter {..} -> Counter {count = count-i, lastModification = (-i)}
    return NoContent
  
  server :: IORef Counter -> Server API
  server ref = getCount ref :<|> addCount ref :<|> subCount ref
  
  main = do
    ref <- newIORef $ Counter 0 0
    run 8080 $ serve (Proxy :: Proxy API) $ server ref
\end{minted}

\end{document}
