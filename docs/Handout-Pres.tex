\documentclass[11pt,utf8,english,hyperref={hidelinks},table]{beamer}

% \usepackage[table]{xcolor}
\usepackage[english]{babel} 				% Deutsche Silbentrennung
\usepackage[utf8]{inputenc}
\usepackage[nopatch=footnote]{microtype}
\usepackage{pgfplots}
\pgfplotsset{compat=1.18}
\usepackage{graphicx}
\usepackage{multicol}
\usepackage{csquotes}
\usepackage{array}

\usepackage{parskip}
\usepackage{comment}
\usepackage[orig,english]{isodate}
\usepackage[bottom]{footmisc}
\usepackage{fancyhdr}
\usepackage[T1]{fontenc}
\usepackage{lmodern}
\usepackage{transparent}

\usepackage{float}
\floatstyle{plain}
\restylefloat{figure}
\usepackage{minted}
\setminted[]{frame=leftline, breaklines, breakbytoken, encoding=utf8, autogobble=true, fontsize=\footnotesize}
\usemintedstyle{xcode}
\makeatletter
\renewcommand{\fps@listing}{htp}
\makeatother

\usepackage{tikz}
\usetikzlibrary{shapes.geometric, calc, shapes.multipart, arrows, matrix, positioning}
\newcommand<>{\btikzset}[2]{\alt#3{\tikzset{#1}}{\tikzset{#2}}}
\tikzset{onslide/.code args={<#1>#2}{%
  \only<#1>{\pgfkeysalso{#2}} % \pgfkeysalso doesn't change the path
}}
\tikzset{temporal/.code args={<#1>#2#3#4}{%
  \temporal<#1>{\pgfkeysalso{#2}}{\pgfkeysalso{#3}}{\pgfkeysalso{#4}} % \pgfkeysalso doesn't change the path
}}

% \usepackage[hidelinks]{hyperref}
\urlstyle{same} % gleiche Schriftart für Hyperlinks.
\hypersetup{
  colorlinks,
  linkcolor={red!50!black},
  citecolor={blue!50!black},
  urlcolor={blue!80!black}
}

\usetheme{Singapore}
\usecolortheme{rose}

\title{Servant - Handout}
\subtitle{A quick run-through of Servant and why}
\author{Cyrill Brunner}
\date{\today}

\begin{document}

  \begin{frame}
    \titlepage
  \end{frame}


  \begin{frame}
    \begin{itemize}[<+->]
      \item Servant is a type-level DSL
      \item Written in Haskell
      \item Encoding as much as possible on the type-level
      \item Enables automatic derivation of:
      \item Server-side function signatures
      \item Client-side accessor functions (in Haskell, but also JS etc.)
      \item Documentation
      \item Swagger/OpenAPI Specification
    \end{itemize}
  \end{frame}

  \begin{frame}
    \begin{itemize}[<+->]
      \item Written to be extensible and pluggable
      \item Ad-hoc extension/wrapping of existing features supported!
      \item Easy introduction of new API parameter/return types -- e.g. just auto-derive ToJSON/FromJSON instances!
      \item Require authentication for an entire section of your server? Just pluck \texttt{Auth '[JWT, Cookie] LoginToken}
            before it
      \item \texttt{LoginToken} not enough? Easy, write a wrapper that queries your database and verifies the token automatically!
      \item And so on...
    \end{itemize}
  \end{frame}

  \begin{frame}
    Check \url{https://docs.servant.dev/} -- The tutorial introduces core concepts, the Cookbook shows
    popular libraries, patterns etc. contributed by the community.
  \end{frame}

  \begin{frame}{Haskell I}
    \begin{itemize}[<+->]
      \item Pure -- No mutation in the language
      \item Lazy -- Evaluation only when necessary
      \item Functional -- Write your program by composing functions
      \item How to \textbf{do} Anything $\rightarrow$ Effects!
    \end{itemize}
  \end{frame}

  \begin{frame}{Haskell II}
    \begin{itemize}[<+->]
      \item \texttt{IO} is the catch-all effect for Input/Output
      \item \texttt{IORef}, \texttt{MVar} -- mutable \enquote{cells} for state
      \item \texttt{Reader} -- Immutable, read-only environment (usable for config)
      \item \texttt{State} -- Mutable environment by passing it along implicitly
      \item \texttt{Except} -- Exceptions and catching them
      \item \texttt{List} -- Usable for non-determinism
      \item \texttt{Writer} -- Collect output (generally not for Logging)
    \end{itemize}
  \end{frame}

  \begin{frame}{Haskell III}
    \begin{itemize}[<+->]
      \item \texttt{RWS} -- ReaderWriterState
      \item \texttt{Accum} -- Append-only version of State/Writer with ability to see previous output
      \item \texttt{Cont} -- Continuations, don't @me
      \item \texttt{Select} -- Selection, for search algorithms, first time I've heard it pleasedon'taskmeIdon'tknow
      \item \texttt{SMT} -- Software Transactional Memory $\rightarrow$ deadlock-free, race-free atomic concurrent state
      \item \texttt{Logging} -- Well, Logging
      \item \textit{Roll your own}
    \end{itemize}
  \end{frame}

  \begin{frame}[fragile]{Example}
    \begin{minted}{haskell}
      {-# LANGUAGE DataKinds #-}
      {-# LANGUAGE DeriveGeneric #-}
      {-# LANGUAGE TypeOperators #-}
      {-# LANGUAGE BlockArguments #-}
      {-# LANGUAGE DeriveAnyClass #-}
      {-# LANGUAGE RecordWildCards #-}
      {-# LANGUAGE DerivingStrategies #-}
      module Main where
      
      import Control.Monad.IO.Class   ( liftIO )
      import Data.Aeson               ( ToJSON, FromJSON )
      import Data.IORef               ( IORef, newIORef
                                      , readIORef, writeIORef
                                      , modifyIORef
                                      )
      import Data.Proxy               ( Proxy )
      import GHC.Generics             ( Generic )
      import Network.Wai.Handler.Warp ( run )
      import Servant.API
      import Servant.Server
    \end{minted}
  \end{frame}

  \begin{frame}[fragile]{Example}
    \begin{minted}{haskell}
      data Counter = Counter { count :: Int, lastModification :: Int }
        deriving stock (Eq, Show, Generic)
        deriving anyclass (FromJSON, ToJSON)
      
      type API =
        -- GET /count, returns { "count": 20, "lastModification": 0 }
        "count" :> Get '[JSON] Counter
        -- POST /add with Int as JSON, returns new Counter
        :<|> "add" :> ReqBody '[JSON] Int :> Post '[JSON] Counter
        -- POST /subtract with Int as JSON, returns nothing
        :<|> "subtract" :> ReqBody '[JSON] Int :> NoContentVerb 'POST
    \end{minted}
  \end{frame}
  
  \begin{frame}[fragile]{Example}
    \begin{minted}{haskell}
      -- Server ("count" :> Get '[JSON] Counter)
      getCount :: IORef Counter -> Handler Counter
      getCount ref = liftIO $ readIORef ref
      
      -- Server ("add" :> ReqBody '[JSON] Int :> Post '[JSON] Counter)
      addCount :: IORef Counter -> Int -> Handler Counter
      addCount ref i = do
        Counter {..} <- liftIO $ readIORef ref
        let counter = Counter { count = count + i
                              , lastModification = i
                              }
        liftIO $ writeIORef ref counter
        return counter
    \end{minted}
  \end{frame}
  
  \begin{frame}[fragile]{Example}
    \begin{minted}{haskell}
      -- Server ("subtract" :> ReqBody '[JSON] Int :> NoContentVerb 'POST)
      subCount :: IORef Counter -> Int -> Handler NoContent
      subCount ref i = do
        liftIO
          $ modifyIORef ref
          $ \Counter {..} -> Counter { count = count-i
                                     , lastModification = (-i)
                                     }
        return NoContent
      
      server :: IORef Counter -> Server API
      server ref = getCount ref :<|> addCount ref :<|> subCount ref
    \end{minted}
  \end{frame}

  \begin{frame}[fragile]{Example}
    \begin{minted}{haskell}
      main :: IO ()
      main = do
        ref <- newIORef $ Counter 0 0
        run 8080 $ serve (Proxy :: Proxy API) $ server ref
    \end{minted}    
  \end{frame}
\end{document}
